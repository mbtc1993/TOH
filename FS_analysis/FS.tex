\documentclass{article}
\usepackage[utf8]{inputenc}

\title{Frame-Stewart}
\author{mb.t.c1993 }
\date{December 2019}

\begin{document}

\maketitle

\section{Defining the Solutions}
\begin{itemize}
    \item Let $H_K(n)$ be the minimal number of moves necessary to move $n$ Hanoi discs with $k$ pegs.

    \item sequence of moves $S$ is a minimal sequence if it has $H_K(n)$ moves

    \item We designate the portion of sequence of $S$ through the triple $(n,\infty,\infty$)  as the demolishing phase, and the portion following the triple through the end as the reconstruction phase. In a minimal sequence we must have the triple $(n,\infty,\infty)$ exactly once.

    \item If $S$ is a minimal solution, its demolishing phase must have length $\frac{H_K(n)+1}{2}$ and its reconstruction phase must have length $\frac{H_K(n)-1}{2}$. This is because a demolishing phase can be used to construct a reconstruction phase, and a reconstruction phase can be used to construct a demolishing phase. For example, given a demolishing phase where the $l$th move is $(n,\infty,\infty)$, for every
    $u<l$ let $(r,s,y)$ be the $l-u$th move, we set the $l+u$th move to be $(r,y,s)$. Such a sequence is called symmetric

    \item Since a minimal demolishing phase can generate a minimal sequence, we may focus on finding minimal demolishing phases.
\end{itemize}

\section{Stacks}
\subsection{Number of Stacks}
\begin{itemize}

    \item Suppose $S$ is a minimal demolishing sequence of $n$ discs with $k$ pegs.
    \item At the end of $S$, that is when we make the move $(n,\infty,\infty)$ we may observe the stacks of discs.
    \item We let $j_1<j_2<...,<j_{k-3}<n-1<n$ denote the stack bottoms.
    \item As long as $n>k-1$ the only empty peg will be the source, as disc $n$ has just moved off of it. N.B. When $n<=k-1$ there is a  
    \item minimal demolishing phase of length $n$ that is easy to calculate.

    \item I will prove that $n>k-1$ implies $k-1$ stacks at the end of a minimal demolishing sequence $S$:
    \item Suppose $S$ is a minimal demolishing sequence.
    \item If some there is some disc $d$ that moves to the empty stack via $(d,d+1,\infty)$ when it is cleared, it must also leave the stack, \item otherwise it wouldn't be empty. So we must have $(d,\inf,x)$ for some $x$, possibly $x$ a stack bottom.
    \item If we remove $(d,\inf,x)$ from $S$ to form $S'$, we see $S'$ is shorter than $S$, hence $S$ can't be minimal. The contradiction implies we have $k-1$ stacks.
    \item We also consider the case where a peg is never occupied at all.
    \item In this case we can derive a minimal demolishing sequence shorter than $S$ by taking $d$ some non-stack bottom, occurring in triples 
    \item $(d,d+1,\inf)$,...,$(d,\inf,x)$ where there are at least two triples since $d$ isn't a stack-bottom. 
    \item Removing all of the above triples from $S$ and instead sending $d$ to the empty stack as soon as it's freed via $(d,d+1,\inf)$ we have constructed a minimal demolishing sequence shorter than $S$. The contradiction implies we must certainly have $k-1$ stacks.
\end{itemize}

\subsubsection{Distribution of Stacks}
\begin{itemize}
\item When $n$ is freed, the $k-1$ stacks are distributed as follows:
\\$j_1,j_1-1,...2,1$
\\$j_2,j_2-1,...j_1+2,j_1+1$
\\$j_3,j_3-1,...j_2+2,j_2+1$
\\....
\\$j_{k-3},j_{k-3}-1,.....j_{k-2}+1$
\\$n-1,n-2,....j_{k-3}+1$
\\$n$
\item To see that this is true, first recall that a disc can only be stacked on a larger disk, so we couldn't have $j_{k-3}$ on $j_1$'s stack, for example.
\item Now suppose $S$ is a minimal demolishing sequence and we have a disk $j_{i-1} \leq d \leq j_i$, the smallest element
on $j_{r}$'s stack, where $r>i$
\item We know $d$'s first move is $(d,d+1,\infty)$
\item We could have $d$ moving before it moves to $j_{r}$'s stack, e.g. $(d,\infty,x),(d,x,x'),(d,x',x'')$ etc.
\item Eventually we have $(d,y,j_{r-1}+1)$ for some $y$
\item We may construct a minimal demolishing sequence $S'$ by sending $d$ to its proper stack as soon as $d+1$ gets pushed onto that         stack, which will free up the later moves to use more pegs and shorten the sequence.
\end{itemize}

\section{Subsequences}
\begin{itemize}
\item For a minimal demolishing sequence $S={(1,2,\infty),(2,3,\infty),...(n,\infty,\infty)}$,
we must have $|S|=\frac{H_K(n)+1}{2}$
\item First we move $j_1$ disks to a non-target peg using k pegs.
\item Let $S_1={S_1^D,S_1^R}={{(1,2,\infty)(2,3,\infty).(j_1,j_1+1,\infty)}{(j_1-1,\inf,j_1)...(1,\infty,2)}}$
\item Since $S$ is a minimal demolishing sequence, $S_1$ must be a minimal sequence of $j_1$ discs using $k$ pegs--if $S_1$ wasn't minimal, we could find $S_1'$ shorter than $S_1$ and use it to contradict minimality of S. Let $S_1^*$ be the symmetric reflection of \item $S_1$ with similar notation for the following subproblems
\item $|S_1|=H_K(j_1)=|S_1^D|+|S_1^R|=\frac{H_K(j_1)+1}{2}+\frac{H_K(j_i)-1}{2}$
\item So $S_1^D$ must be a minimal demolishing sequence of $j_1$ disks using $k$ pegs. If $|S_1|$ isn't symmetric, we can set $S_1^R$ as the symmetric reflection of $s_1^D$ to generate a minimal symmetric sequence. We see that existence of a minimal demolishing sequence ensures existence of a minimal symmetric sequence.
\item Also note that when we generate the entire $n$ disk demolishing and reconstructing solution we will calculate the reconstruction phase by symmetry. In the reconstruction phase, moving the $j_1$ disks to the target peg using $k$ pegs is the last step.
\item Next we move $j_2-j_1$ disks using $k-1$ pegs
\item We can define $S_2={S_2^D,S_2^R}$. Analogous to the above subproblem, $S_2^D$ must be a minimal demolishing sequence, we use symmetry to get the reconstruction phase, and let $S_2^*$ denote the symmetric reflection of $S_2$
\item Moving $(n-1,n-2,....j_{k-3}+1)$ using $3$ pegs generates $S_{k-2}$
\item Moving the bottom disk with $k=2$ gives $S_{k-1}$
\item We can see $(S,S*)=(S_1,S_2,..,S_{k-2},S_{k-1},S_{k-2},S_{k-3},S_{k-2}*,...,S_2*,S_1*)$
\end{itemize}
$$|(S,S*)|=H_K(n)=1+2H_K(j_1)+2*H_{K-1}(j_2-j_1)+$$
$$2*H_{K-2}(j_3-j_2)+\dots+$$
$$2*H_{4}(j_{k-3}-j_{k-2})+$$
$$2*H_{3}(n-1-j_{k-3})$$

\section{Complexity}
\begin{itemize}
\item Next define sufficient symmetry conditions, and demonstrate why the example in Hinz et al.is degenerate
\item Derive the optimal bound

\end{itemize}

\end{document}
